%
% Copyright (c) 2017 Radoslaw Kujawa.
% All rights reserved.
%
% Redistribution and use in source and binary forms, with or without
% modification, are permitted provided that the following conditions
% are met:
%
% 1. Redistributions of source code must retain the above copyright
%    notice, this list of conditions and the following disclaimer.
% 2. Redistributions in binary form must reproduce the above copyright
%    notice, this list of conditions and the following disclaimer in the
%    documentation and/or other materials provided with the distribution.
%
% THIS SOFTWARE IS PROVIDED BY RADOSLAW KUJAWA (THE AUTHOR) AND CONTRIBUTORS
% ``AS IS'' AND ANY EXPRESS OR IMPLIED WARRANTIES, INCLUDING, BUT NOT LIMITED
% TO, THE IMPLIED WARRANTIES OF MERCHANTABILITY AND FITNESS FOR A PARTICULAR
% PURPOSE ARE DISCLAIMED.  IN NO EVENT SHALL THE AUTHOR OR CONTRIBUTORS
% BE LIABLE FOR ANY DIRECT, INDIRECT, INCIDENTAL, SPECIAL, EXEMPLARY, OR
% CONSEQUENTIAL DAMAGES (INCLUDING, BUT NOT LIMITED TO, PROCUREMENT OF
% SUBSTITUTE GOODS OR SERVICES; LOSS OF USE, DATA, OR PROFITS; OR BUSINESS
% INTERRUPTION) HOWEVER CAUSED AND ON ANY THEORY OF LIABILITY, WHETHER IN
% CONTRACT, STRICT LIABILITY, OR TORT (INCLUDING NEGLIGENCE OR OTHERWISE)
% ARISING IN ANY WAY OUT OF THE USE OF THIS SOFTWARE, EVEN IF ADVISED OF THE
% POSSIBILITY OF SUCH DAMAGE.
%
% 
\documentclass[dvipsnames,table]{beamer}
\usepackage{polski}

\usetheme{Rochester}
\usecolortheme{orchid}

\usepackage{listings}
\usepackage{ucs}
\usepackage[utf8x]{inputenc}
\usepackage{wasysym}
\usepackage[normalem]{ulem}
\usepackage{amsmath}
\usepackage{hyperref}
\usepackage{tikzsymbols}

\setbeamertemplate{navigation symbols}{}
\setbeamertemplate{caption}[numbered]
\setbeamerfont{caption}{size=\scriptsize}
\setbeamercolor{framenote}{bg=OSEC-red!25}
\setbeamercolor{rednote}{bg=Red!25}
\setbeamercolor{palette primary}{use=structure,fg=white,bg=OSEC-red}
\setbeamercolor{palette secondary}{use=structure,fg=white,bg=OSEC-red2}

\setbeamertemplate{itemize item}{\scriptsize\raise1pt\hbox{\donotcoloroutermaths$\blacktriangleright$}}
\setbeamertemplate{itemize subitem}{\tiny\raise1pt\hbox{\donotcoloroutermaths$\bullet$}}
\setbeamertemplate{itemize subsubitem}{\tiny\raise1pt\hbox{\donotcoloroutermaths{--}}}

\setbeamertemplate{enumerate item}{\insertenumlabel.}
\setbeamertemplate{enumerate subitem}{\insertenumlabel.\insertsubenumlabel}
\setbeamertemplate{enumerate subsubitem}{\insertenumlabel.\insertsubenumlabel.\insertsubsubenumlabel}
\setbeamertemplate{enumerate mini template}{\insertenumlabel}

\setbeamercolor{itemize item}{fg=OSEC-red, bg=OSEC-red}
\setbeamercolor{itemize subitem}{fg=OSEC-red, bg=OSEC-red}
\setbeamercolor{itemize subsubitem}{fg=OSEC-red, bg=OSEC-red}

\setbeamercolor{section number projected}{fg=white,bg=OSEC-red}
\setbeamercolor{subsection number projected}{fg=white,bg=OSEC-red}
\setbeamercolor{button}{bg=OSEC-red,fg=white}

\setbeamertemplate{section in toc}[circle]
\setbeamertemplate{subsection in toc}[square]

\definecolor{OSEC-red}{RGB}{160,29,44}
\definecolor{OSEC-red2}{RGB}{177,76,12}
\hypersetup{colorlinks=true,linkcolor=white,urlcolor=OSEC-red}

\setlength{\tabcolsep}{8pt}
\renewcommand{\arraystretch}{1.2}

\newcommand{\tri}{$\triangleright$ }

\title{OpenStack Neutron -- Software Defined Networking w prywatnych chmuarch}
\author{Radosław Kujawa -- radoslaw.kujawa@osec.pl}
\institute{OSEC}

\begin{document}

\begin{frame}
	\titlepage
\end{frame}

\begin{frame}
\frametitle{OpenStack + Neutron}
\begin{itemize}
	\item OpenStack -- nie tylko wirtualizacja CPU i pamięci, ale też pamięci masowej i sieci.
	\item Neutron -- zarządzanie infrastrukturą sieciową L2/L3 dla potrzeb chmury.
	\item Różne potrzeby klientów -- konieczność stworzenia bardzo elastycznego mechanizmu wirtualizacji sieci.
	\item Współpraca z zewnętrzną infrastrukturą sieciową.
\end{itemize}
\end{frame}

\begin{frame}
\frametitle{Funkcjonalności Neutrona}
\begin{itemize}
	\item ML2 
	\item Layer 3 agents (routing, DHCP).
	\item Foo as a Service
	\begin{itemize}
		\item Firewall as a Service.
		\item VPN as a Service.
		\item Load balancer as a Service.
		\item \dots 
	\end{itemize}
\end{itemize}
\end{frame}

\begin{frame}
\frametitle{Architektura sieci w Neutronie}
\begin{itemize}
	\item Założenie: każdy projekt w OpenStack może posiadać zupełnie odrębne, niezależne sieci (oddzielna warstwa 2).
	\item Projekty w OpenStack mogą o sobie na wzajem nic nie wiedzieć.
\end{itemize}
\end{frame}

\begin{frame}
\frametitle{Sieci ,,tenant''}
\begin{itemize}
	\item Domyślna architektura idealna pod potrzeby ,,hostingowe''. 
	\item Izolacja projektów -- tunelowanie sieci prywatnych
	\item Jeden punkt styku z siecią fizyczną (na {\tt network node}).
	\item Instancje maszyn wirtualnych {\em nie mogą} posiadać interfejsów przyłączonych bezpośrednio do sieci fizycznej.
\end{itemize}
\end{frame}

\begin{frame}
\frametitle{Sieci ,,provider''}
\begin{itemize}
	\item Alternatywny sposób konfiguracji Neutrona.
	\item Sieci {\tt flat} -- bezpośredni dostęp maszyny wirtualnej do sieci fizycznej.
	\item Separacja sieci projektowych w dalszmy ciągu możliwa:
	\begin{itemize}
		\item VLAN
		\item VXLAN
		\item Geneve
		\item \dots
	\end{itemize}
\end{itemize}
\end{frame}

\begin{frame}
\frametitle{Integracja Neutrona z zewnętrznymi rozwiązaniami}
\begin{itemize}
	\item API REST oferowane przez Neutron.
	\item Pluginy do zarządzania warstwą 2 i 3.
	\begin{itemize}
		\item Cisco Nexus.
		\item Juniper.
		\item VMware NSX.
	\end{itemize}
	\item Integracja z kontrolerami SDN i NFV.
	\item OpenContrail.
\end{itemize}
\end{frame}

\begin{frame}
\frametitle{Demo}
\begin{itemize}
	\item Demo: utworzenie sieci zewnętrznych.
	\item Demo: przyłączenie maszyny wirtualnej bezpośrednio do sieci zewnętrznej.
	\item Demo: utworzenie sieci prywatnej.
	\item Demo: tworzenie routera między siecią prywatną i zewnętrzną.
	\item Demo: przyłączenie maszyny wirtualnej do sieci prywatnej.
\end{itemize}
\end{frame}

\begin{frame}
\frametitle{Koniec\ldots}
\begin{center}
\includegraphics[scale=0.5]{img-oseclogo.png}

Dziękuje!

Czy są pytania?

\end{center}
\end{frame}
\end{document}

